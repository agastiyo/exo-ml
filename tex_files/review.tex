\documentclass[preprint,longauthor]{aastex631}

\begin{document}

\title{Literature Review: Influence of Host Star Properties on Exoplanet Occurrence Rates}
\author{Agastya Gaur}
\affil{University of Illinois at Urbana-Champaign}

\begin{abstract}
  This review examines how host star properties influence exoplanet occurrence rates, focusing on metallicity, rotation rate, age, and mass. Metallicity shows the strongest correlation, with higher-metallicity stars more likely to host giant planets, consistent with core accretion theory. Smaller planets exhibit weaker or no metallicity dependence. Correlations involving rotation rate and age remain inconclusive due to limited and heterogeneous data, while stellar mass appears positively correlated with giant planet occurrence but inversely related for smaller planets. Further work is needed to build a clear statistical picture of exoplanet occurrence. Understanding these correlations helps constrain models of planet formation and guide target selection for future surveys.
\end{abstract}

\section{Introduction}

Since the first discovery of exoplanets by \citet{mayorJupitermassCompanionSolartype1995}, over 6,000 unique exoplanets have been discovered. From these discoveries, patterns in the characteristics of exoplanet host (EH) stars have been identified. \citet{gonzalezStellarMetallicityGiant1997} first noted that stars hosting Jupiter-mass exoplanets tend to have higher metallicities compared to stars without detected planets, though the same trend was not observed in lower mass exoplanets \citep{udryHARPSSearchSouthern2006}. \citet{barnesAssessmentRotationRates2001} suggested that stellar rotation rates may also influence planet formation, with faster rotating stars being more likely to host massive planets, though the heterogeneity of the data made the study inconclusive. \citet{lawsReevaluationSuperLITHIUMrichStar2003} found EH stars tended to be older than non-EH stars, though the data was highly uncertain.

Overall, metallicity appears to be the most robustly correlated stellar property with EH frequency, though other properties such as rotation rate and age may also play a role. Despite these correlations, predictive models of planet occurrence based on stellar properties remain uncertain due to limited samples and detection biases.

The following literature section reviews key studies that have investigated the influence of EH star properties on exoplanet occurrence rates, summarizing their methodologies, findings, and limitations. The paper is organized into subsections focusing on different stellar properties.

\section{Literature}

A comprehensive overview of stellar correlations with exoplanet occurrence is provided by \citet{udryStatisticalPropertiesExoplanets2007} and \citet{zhuExoplanetStatisticsTheoretical2021a}. The authors summarize findings on the correlations between EH star properties and exoplanet occurrence rates, highlighting metallicity as a key factor. They also cover a wide range of other stellar properties, including mass, age, and rotation rate, discussing their potential influence on planet formation and detection biases. The authors note the limitations of existing studies, including small sample sizes and observational biases.

\subsection{Metallicity}

The accretion disk model of planet formation predicts that higher metallicity stars should have a higher likelihood of hosting giant planets, as the increased abundance of heavy elements facilitates the formation of solid cores necessary for gas giant formation \citep{udryStatisticalPropertiesExoplanets2007,zhuExoplanetStatisticsTheoretical2021a}. Smaller planets, however, may not show the same strong correlation with metallicity, as they do not need solid cores as massive as those required for giant planet formation. Theoretically, strong correlations between stellar metallicity and giant planet occurrence rates are expected, while smaller planets may exhibit weaker or no correlation.

\citet{gonzalezStellarMetallicityGiant1997} analyzed a sample of stars with and without detected planets, finding that EH stars with hot Jupiter-type exoplanets had significantly higher metallicities on average. Furthermore, the study found that the metallicity enhancement was more pronounced for stars hosting giant planets compared to those hosting smaller planets. \citet{buchhaveAbundanceSmallExoplanets2012} expanded on this by analyzing a larger sample of stars hosting small exoplanets, defined as planets with radii less than 4 earth radii. They found that while giant planet hosts tend to be metal-rich, stars hosting smaller planets do not show a significant metallicity enhancement compared to non-EH stars.

\citet{udryHARPSSearchSouthern2006} further postulated that hot Neptunes and super-Earths may not follow the same metallicity trend as giant planets. Their analysis of a sample of stars hosting these smaller exoplanets showed no significant difference in metallicity compared to stars without detected planets.

Overall, these studies consistently found that stellar metallicity is a key factor influencing the occurrence rates of giant exoplanets, while smaller planets appear less sensitive to host star metallicity. All three studies attributed this trend to the core accretion model of planet formation, where higher metallicity environments facilitate the rapid formation of solid cores necessary for giant planet formation, while smaller planets can form in a wider range of metallicity environments. They also acknowledged the limitations of their samples and potential biases introduced by detection methods.

\subsection{Rotation}

\citet{barnesAssessmentRotationRates2001} investigated the relationship between stellar rotation rates and the presence of giant exoplanets. The authors analyzed a sample of EH stars and compare their rotation rates to a control sample of non-EH stars. They found that EH stars tend to have faster rotation rates, suggesting a potential link between stellar rotation and planet formation.

EH stars exhibit lower lithium abundances compared to non-EH stars, potentially due to enhanced lithium depletion caused by the presence of planets \citep{israelianEnhancedLithiumDepletion2009}. \citet{eggenbergerEffectsRotationMagnetic2010} explored the impact of rotation rate on the lithium abundance of stars. The authors suggested that faster rotating stars may experience stronger magnetic activity, leading to increased lithium depletion, which could be linked to the presence of planets.

While \citet{barnesAssessmentRotationRates2001} suggested faster rotation may correlate with planet presence, later work connected rotation-driven mixing and lithium depletion \citep{eggenbergerEffectsRotationMagnetic2010,israelianEnhancedLithiumDepletion2009} as possible indirect indicators of planet-star interaction. Both studies suggested a potential correlation between stellar rotation rates and the presence of exoplanets, though the exact nature of this relationship remains unclear. They acknowledged the limitations of their samples and the need for further research to confirm these findings.

\subsection{Age}

Given the fact that older stars have had a longer time to form exoplanets, a slight positive correlation between stellar age and exoplanet occurrence can be expected.

\citet{lawsReevaluationSuperLITHIUMrichStar2003} examined the ages of EH stars compared to non-EH stars. The authors found that EH stars tend to be older on average, suggesting that planet formation may be more likely around older stars.

\citet{meloAgeStarsHarboring2006} built on this theory by analyzing a larger sample of EH stars that host transiting planets. They found a wide range of ages among EH stars, indicating no correlation between stellar age and exoplanet occurrence. However, the study was able to place a lower bound on the age of EH stars at 1 Gyr, suggesting that very young stars are less likely to host planets.

\citet{sayeedExoplanetOccurrenceRate2025} utilized data from the Kepler mission to analyze the occurrence rates of exoplanets around stars of different ages. The authors found a statistically insignificant correlation between stellar age and exoplanet ocurrence.

Overall, these studies suggested that while there may be a tendency for EH stars to be older, planet formation can occur around stars of various ages. The exact relationship between stellar age and exoplanet occurrence rates remains uncertain, with limitations in sample sizes and age determination methods acknowledged by the authors.

\subsection{Mass}

Theoretically, stellar mass is corellated with the mass of the protoplanetary disk, which allows for more material for planet formation \citep{zhuExoplanetStatisticsTheoretical2021a}. Stars with higher mass would then be expected to host more massive planets or a higher number of planets.

\citet{johnsonGiantPlanetOccurrence2010} investigated the relationship between stellar mass and the occurrence rates of giant exoplanets. The authors analyzed a sample of EH stars with varying masses and found a roughly linear correlation. On the other hand, \citet{muldersINCREASEMASSPLANETARY2015} found that low mass stars tended to host more small planets, suggesting that while giant planet occurrence increases with stellar mass, smaller planets may be more common around lower mass stars. The authors suggested that low mass stars are more efficient at converting protoplanetary disk mass to planets, though they could not provide an explanation why.

Overall, stellar mass both studies showed that stellar mass was positively correlated with the occurrence of large exoplanets, while maintaining a possible negative correlation with smaller planets.

\section{Conclusion}

The literature indicates that stellar properties strongly influence exoplanet occurrence rates. Stellar metallicity emerges as the most robustly correlated property, particularly for giant exoplanets, while smaller planets appear less sensitive to host star metallicity. The potential roles of stellar rotation rates and ages are also explored, though findings remain inconclusive due to limited sample sizes and observational biases. Further research with larger, more diverse samples and improved detection methods is necessary to clarify these relationships and enhance our understanding of planet formation.

In order to better classify stars as potential exoplanet hosts, future studies should focus on refining the correlations between stellar properties and exoplanet occurrence rates. Machine learning techniques could be employed to analyze large datasets from missions like Kepler and TESS to identify and weight the most predictive stellar characteristics. This can help inform more robust models for predicting exoplanet occurrence rates based on host star properties, building an accurate statistical picture of the likelihood of stars being EH stars. Such models can improve the efficiency of exoplanet detection efforts.

\bibliography{references}

\end{document}